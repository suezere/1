首先推导了无人机绳系吊运系统的数学模型。从单个无人机出发,基于牛顿-欧拉法建立了单个四旋翼无人机的数学模型,并通过拉格朗日法推导出了单个无人机吊装载荷的混合模型。对于多无人机绳系吊运系统,在单无人机绳系吊运系统的基础上,基于拉格朗日方程建立了系统的模型,并进行了简化。

由于绳子和载荷的存在,特别是在机动过程中系绳上会存在连续变化的干扰力,无人机绳系吊运系统难以实现高精度的路径跟踪控制。针对上述问题,受Koopman算子理论启发,提出了一种基于数据驱动的建模方法(Neural Predictor),用于无人机在激进飞行条件下的绳系吊挂载荷控制。Neural Predictor将有效载荷及无人机自身空气动力产生的外力和力矩等不确定非线性项干扰建模为一个升维的线性动态系统,设计相应的损失函数通过深度神经网络进行预测学习,并从理论上保证了预测误差的有界性,显著提高了预测模型的准确性;将学习得到的动力学模型与无人机的名义动力学系统相结合形成了完整的系统混合模型,集成到模型预测控制框架中,提高了系统在未知扰动下的鲁棒性。

多个无人机在协同吊运载荷时,需要综合考虑无人机的三维运动和欠驱动特性,并关注单体无人机之间的信息交互与干扰,系绳和载荷的存在进一步加剧了系统的耦合复杂性,因此系统的参数通常需要大量的手动调整。针对上述问题,在分布式模型预测控制方法的基础上,提出了一种新颖的自适应调参方法(Autotune),以闭环方式有效地学习模型预测控制中的权重参数,从而提高了多无人机绳系吊运系统的协调控制性能。Autotune引入分布式灵敏度传播算法并行计算无人机上的灵敏度,在此基础上结合深度神经网络设
计了分布式策略梯度学习算法生成自适应归一化超参数,使控制器能够动态适应不同载荷和环境变化,提高了多无人机绳系吊运系统的协调控制性能。